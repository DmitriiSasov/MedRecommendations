\documentclass{article}%
\usepackage[T1]{fontenc}%
\usepackage[utf8]{inputenc}%
\usepackage{lmodern}%
\usepackage{textcomp}%
\usepackage{lastpage}%
\usepackage[tmargin=2cm,lmargin=1cm]{geometry}%
\usepackage[russian]{babel}%
\usepackage{tikz}%
\usepackage{alltt}%
%
\usepackage{times}%
%
\begin{document}%
\normalsize%
\section*{Оглавление}%
\label{sec:}%
\begin{itemize}%
\item%
Диагностика%
\item%
Лечение%
\item%
Критерии оценки качества медицинской помощи%
\end{itemize}

%
\newpage%
\section{Диагностика}%
\label{sec:}%
\subsection{Физикальное обследование}%
\label{subsec:}%
\begin{itemize}%
\item%
Всем пациентам с АГ рекомендуется пальпировать пульс в покое для измерения его частоты и ритмичности с целью выявления аритмий {[}21; 32; 43{]}.%
\newline%
ЕОК/ЕОАГ (УУР В, УДД 2) = В2%
\item%
Всем пациентам с АГ рекомендуется определение антропометрических данных для выявления избыточной массы тела/ожирения, оценка неврологического статуса и когнитивной функции, исследование глазного дна для выявления гипертонической ретинопатии, пальпация и аускультация сердца и сонных артерий, пальпация и аускультация периферических артерий для выявления патологических шумов, сравнение АД между руками хотя бы однократно {[}21{]}.%
\newline%
ЕОК/ЕОАГ нет (УУР С, УДД 5) = %
\begin{tikzpicture}%
\node[draw,align=center,fill=orange] (box) {С};%
\node[draw,align=center,fill=blue] (box) {5};%
\end{tikzpicture}%
\end{itemize}

%
\subsection{Лабораторное обследование}%
\label{subsec:}%
\begin{itemize}%
\item%
Всем пациентам с АГ для выявления гиперурикемии рекомендуется исследование уровня мочевой кислоты в крови {[}71{]}.%
\newline%
ЕОК/ЕОАГ нет (УУР А, УДД 2) = А2%
\item%
Всем пациентам с АГ для выявления нарушения функции почки оценки сердечно{-}сосудистого риска рекомендуются исследование уровня креатинина в сыворотке крови и расчет скорости клубочковой фильтрации (СКФ) в мл/мин/1,73м2 по формуле Chronic Kidney Disease Epidemiology (CKD{-}EPI) {[}58{]} в специальных калькуляторах (Таблица П3, Приложение Г3) {[}21, 22, 58{]}.%
\newline%
ЕОК/ЕОАГ IВ (УУР B, УДД 2) = В2%
\item%
Всем пациентам с АГ для выявления заболеваний почек и оценки СС риска рекомендуется проводить общий (клинический) анализ мочи с микроскопическим исследованием осадка мочи, количественной оценкой альбуминурии или отношения альбумин/креатинин (оптимально) {[}64, 65{]}.%
\newline%
ЕОК/ЕОАГ IВ (УУР В, УДД 2) = В2%
\item%
Всем пациентам с АГ для стратификации риска и выявления нарушений липидного обмена рекомендуется исследование уровня общего холестерина (ОХС), холестерина липопротеинов высокой плотности (ХС{-}ЛВП), холестерина липопротеинов низкой плотности (ХС{-}ЛНП) (прямое измерение или расчетно) и триглицеридов (ТГ) в крови {[}21, 67, 68{]}.%
\newline%
ЕОК/ЕОАГ IВ (УУР В, УДД 2) = В2%
\item%
Всем пациентам с АГ с целью исключения вторичной гипертензии рекомендуется проведение общего (клинического) анализа крови (гемоглобин/гематокрит, лейкоциты, тромбоциты) {[}21, 22{]}.%
\newline%
ЕОК/ЕОАГ нет (УУР C, УДД 5) = С5%
\item%
Для выявления предиабета, СД и оценки сердечно{-}сосудистого риска всем пациентам с АГ рекомендуется исследование уровня глюкозы в венозной крови {[}53,54,55,56,57, 302{]}.%
\newline%
ЕОК/ЕОАГ нет (УУР C, УДД 5) = С5%
\item%
Всем пациентам с АГ для выявления электролитных нарушений и дифференциального диагноза с вторичной АГ рекомендуется исследование уровня калия и натрия в крови {[}21, 22{]}.%
\newline%
ЕОК/ЕОАГ нет (УУР C, УДД 5) = С5%
\end{itemize}

%
\subsection{Инструментальная диагностика}%
\label{subsec:}%
\begin{itemize}%
\item%
Пациентам с АГ при наличии неврологических симптомов и/или когнитивных нарушений рекомендуется выполнение КТ или МРТ головного мозга для исключения инфарктов мозга, микрокровоизлияний и повреждений белого вещества и других патологических образований {[}21, 91, 92{]}.%
\newline%
ЕОК/ЕОАГ IIаВ (УУР А, УДД 1) = А1%
\item%
Всем пациентам с АГ для выявления ГЛЖ и определения СС риска рекомендуется проведение 12{-}канальной ЭКГ {[}21, 22, 78, 297{]}.%
\newline%
ЕОК/ЕОАГ IВ (УУР В, УДД 1) = В1%
\item%
Пациентам с АГ в сочетании с ЦВБ или признаками атеросклеротического поражения сосудов других локализаций, при указании в анамнезе на преходящую слабость в конечностях с одной стороны или онемение половины тела, а также мужчинам старше 40 лет, женщинам старше 50 лет и пациентам с высоким общим сердечно{-}сосудистым риском (Таблица П12, Приложение Г2) рекомендуется дуплексное сканирование брахиоцефальных артерий для выявления атеросклеротических бляшек/стенозов внутренних сонных артерий {[}21, 298{]}.%
\newline%
ЕОК/ЕОАГ IВ (УУР В, УДД 1) = В1%
\item%
Всем пациентам с нарушением функции почек, альбуминурией и при подозрении на вторичную АГ рекомендуется проведение УЗИ (ультразвукового исследования) почек и дуплексного сканирования артерий почек с целью оценки размеров, структуры, а также наличия врожденных аномалий почек или стеноза почечных артерий {[}60, 61, 64{]}.%
\newline%
ЕОК/ЕОАГ IIaC (УУР В, УДД 1) = В1%
\item%
Пациентам с АГ при наличии изменений на ЭКГ или симптомов/признаков дисфункции левого желудочка рекомендуется проведение ЭхоКГ для выявления степени ГЛЖ {[}21, 22, 81{]}.%
\newline%
ЕОК/ЕОАГ IВ (УУР B, УДД 2) = В2%
\item%
Рекомендуется определение ЛПИ в целях уточнения категории риска пациентам с симптомами значимого атеросклероза артерий нижних конечностей или пациентам умеренного риска,  у которых положительные результаты данного исследованияприведут к изменению категории риска {[}86, 87{]}.%
\newline%
ЕОК/ЕОАГ IIbВ (УУР B, УДД 2) = В2%
\item%
Пациентам с АГ 2–3{-}й степеней, всем пациентам с сахарным диабетом и АГ рекомендуется проводить исследование глазного дна врачом{-}офтальмологом (геморрагии, экссудаты, отек соска зрительного нерва) для выявления гипертонической ретинопатии {[}21, 89{]}.%
\newline%
ЕОК/ЕОАГ IС (УУР С, УДД 4) = С4%
\end{itemize}

%
\subsection{Иные диагностические исследования}%
\label{subsec:}%
\begin{itemize}%
\item%
Когнитивные нарушения у пожилых пациентов частично ассоциированы с АГ, в связи с чем у пожилых пациентов с анамнезом, позволяющим предположить ранний когнитивный дефицит, рекомендована оценка когнитивной функции с использованием теста MMSE (MiniMentalStateExamination) {[}93, 94{]}.%
\newline%
ЕОК/ЕОАГ нет (УУР A, УДД 1) = А1%
\end{itemize}

%
\newpage%
\section{Лечение}%
\label{sec:}%
\subsection{Медикаментозная терапия АГ}%
\label{subsec:}%
\subsection{Общие принципы медикаментозной терапии}%
\label{subsec:}%
\begin{itemize}%
\item%
Всем пациентам с АГ (кроме пациентов низкого риска с АД<150/90 мм рт. ст., пациентов ≥80 лет, пациентов с синдромом старческой астении) в качестве стартовой терапии рекомендована комбинация антигипертензивных препаратов, предпочтительно фиксированная, для улучшения приверженности к терапии. Предпочтительные комбинации должны включать блокатор ренин{-}ангиотензиновой системы (РААС) (ингибитор АПФ или БРА) и дигидропиридиновый АК или диуретик (Приложение Б2) {[}130–134{]}.%
\newline%
ЕОК/ЕОАГ IA (УУР А, УДД 1) = А1%
\item%
Пациентам с АГ, не достигшим целевого АД на фоне тройной комбинированной терапии, рекомендуется добавление спиронолактона (подробнее в разделе 3.6.11.) {[}106, 137, 138, 169{]}.%
\newline%
ЕОК/ЕОАГ IB (УУР А, УДД 1) = А1%
\item%
Всем пациентам с АГ не рекомендуется назначение комбинации двух блокаторов РААС вследствие повышенного риска развития гиперкалиемии, гипотензии и ухудшения функции почек {[}21, 139, 145, 146{]}.%
\newline%
ЕОК/ЕОАГ IIIА (УУР A, УДД 1) = А1%
\item%
Пациентам, не достигшим целевого АД на фоне двойной комбинированной терапии, рекомендуется тройная комбинация, как правило, блокатора РААС с АК и тиазидовым/тиазидоподобным диуретиком, предпочтительно в форме фиксированной комбинации {[}135, 136{]}.%
\newline%
ЕОК/ЕОАГ IА (УУР В, УДД 1) = В1%
\end{itemize}

%
\subsection{Основные классы препаратов для лечения артериальной гипертензии}%
\label{subsec:}%
\begin{itemize}%
\item%
У пациентов, не достигших целевого АД при приеме моно{-} или комбинированной АГТ, не включавшей диуретики, рекомендуется назначение низких доз тиазидных или тиазидоподобных диуретиков в составе комбинированной терапии с БРА, ИАПФ и АК для усиления АГЭ и достижения целевого АД {[}150–152{]}.%
\newline%
ЕОК/ЕОАГ IB (УУР А, УДД 1) = А1%
\item%
Альфа{-}адероноблокаторы рекомендуются при резистентной АГ (подробнее в разделе 3.6.11.), в качестве четвертого препарата к комбинации ИАПФ/БРА, АК, диуретика (при непереносимости спиронолактона**) {[}137{]}.%
\newline%
ЕОК/ЕОАГ нет (УУР B, УДД 2) = В2%
\item%
Моксонидин для лечения АГ рекомендуется пациентам с МС или ожирением в комбинации с ИАПФ, БРА, АК и диуретиками при недостаточной эффективности классических комбинаций {[}154–156{]}.%
\newline%
ЕОК/ЕОАГ нет (УУР B, УДД 3) = В3%
\item%
ББ рекомендованы в качестве антигипертензивной терапии при наличии особых клинических ситуаций: например, стенокардии, перенесенного инфаркта миокарда, сердечной недостаточности {[}21, 22{]}.%
\newline%
ЕОК ЕОАГ IA (УУР С, УДД 5) = С5%
\end{itemize}

%
\newpage%
\section{Критерии оценки качества медицинской помощи}%
\label{sec:}%
\subsection{Медикаментозная терапия АГ}%
\label{subsec:}%
Типичными дефектами при оказании медицинской помощи пациентам с АГ являются:%
\begin{alltt}
\underline{при сборе анамнеза:}
\end{alltt}%
\begin{itemize}%
\item%
не уточнены характер начала заболевания, продолжительность, особенности течения заболевания;%
\item%
отсутствуют сведения об эффективности ранее проводимой терапии, о возможном приеме пациентами других, помимо антигипертензивных, лекарственных препаратов: глюкокортикоидных гормонов, цитостатиков, нестероидных противовоспалительных препаратов, оральных контрацептивов и др.%
\item%
отсутствуют сведения о наличии менопаузы у женщин, характере питания, статусе курения, семейном анамнезе ранних сердечно{-}сосудистых заболеваний, осложнений АГ;%
\item%
отсутствие сведений о наличии предшествующих госпитализаций.%
\end{itemize}%
\newline%
\begin{alltt}
\underline{при обследовании пациентов:}
\end{alltt}%
\begin{itemize}%
\item%
неполное физическое, лабораторное и инструментальное обследование, что приводит к недооценке возможности наличия симптоматической АГ, неверной оценке ПОМ и СС риска;%
\end{itemize}%
\newline%
\begin{alltt}
\underline{при постановке диагноза:}
\end{alltt}%
\begin{itemize}%
\item%
отсутствие развернутого клинического диагноза, с указанием стадии гипертонической болезни, степени повышения АД (степени АГ при впервые выявленной АГ), с максимально полным отражением ФР, ПОМ, ССЗ, ХБП и категории сердечно{-}сосудистого риска;%
\item%
необоснованное и неверное установление стадии ГБ и степени АГ, категории риска;%
\item%
отсутствие сведений о наличии у пациента ПОМ, сопутствующих заболеваний и факторов риска;%
\end{itemize}%
\newline%
\begin{alltt}
\underline{при проведении лечения:}
\end{alltt}%
\begin{itemize}%
\item%
измерение АД на высоте эффекта АГТ;%
\item%
назначение нерациональных комбинаций АГП, в неверном режиме и отсутствие интенсификации антигипертензивной терапии;%
\item%
недооценка наличия сопутствующей патологии, влияющей на выбор антигипертензивной терапии;%
\end{itemize}%
\newline%
\begin{alltt}
\underline{при обеспечении преемственности:}
\end{alltt}%
\begin{itemize}%
\item%
отсутствие назначения повторных визитов для контроля АД;%
\item%
несвоевременная постановка на диспансерный учет;%
\item%
нерегулярность диспансерных осмотров.%
\end{itemize}

%
\end{document}